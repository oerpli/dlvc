% !TeX root = ./DLVCQuestions.tex
\section{Image classification}
\subsection{What is the task definition of image classification?}
Predict the label of an image, i.e.: what is visible on the image. Output could also be a distribution over labels to indicate confidence in result.

\subsection{Explain at least 5 challenges and give examples.}
\begin{itemize}
\item Viewpoint variation: A single instance of an object can be oriented in many ways with respect to the camera.
\item Scale variation: Visual classes often exhibit variation in their size (size in the real world, not only in terms of their extent in the image).
\item Deformation: Many objects of interest are not rigid bodies and can be deformed in extreme ways, e.g. water
\item Occlusion: Only a small portion of an object (as little as few pixels) could be visible.
\item Illumination conditions: The effects of illumination are drastic on the pixel level.
\item Background clutter: The objects of interest may blend into their environment, making them hard to identify.
\item Intra-class variation: The classes of interest can often be relatively broad, such as chair. There are many different types of these objects, each with their own appearance.
\item Abstraction (human able to recognize some doodle as the correct object usually---difficult to recognize for machines as e.g. difference between cow and dog in a doodle may be the presence of grass
\item Invariants (an object may be the same object if it's rotated by some degree. Other objects (e.g. ``6'' and ``9'') are not invariant under rotation
\end{itemize}

\subsection{What is object detection and how does it differ from classification?}
Object detection tries to find all objects present in an image whereas in image classification a label is assigned to the whole image. Object detection can be reduced to image classification (classify various parts of the image and combine results)
