% !TeX root = ./DLVCQuestions.tex
\section{Preprocessing}
\subsection{Explain the purpose of preprocessing.}
\subsection{How do per-sample normalization and per-trainingset normalization differ in terms of operation and purpose?}
\subsection{In the latter case, which (if any) preprocessing is applied during validation and testing?}

\section{Optimization vs ML}
\subsection{What are the goals of optimization and machine learning?}
In optimization the goal is to find an optimal value for a function, 
\subsection{Why do they differ?}
\subsection{Create two sketches with each showing the training progress over time in terms of both training and test error; one that is good from an optimization perspective but bad from a machine learning perspective, and one that is worse from an optimization perspective but better from a machine learning perspective. Explain both sketches.}

\section{Regularization}
\subsection{What is the purpose of regularization?}
\subsection{What is weight decay and what is its purpose?}
\subsection{What is early stopping and how does it work?}
\subsection{In deep learning, is it better to increase regularization or to decrease the model capacity by other means, and why?}

\section{CNN backends}
\subsection{What is the backend of a CNN?}
\subsection{Discuss the backends of VGGNet and GoogLeNet/ResNet and their pros and cons.}

\section{Batch normalization}
\subsection{What is the purpose and aim of batch normalization?}
\subsection{Why does it have a regularizing effect?}
\subsection{To which CNN layers is batch normalization applied and where?}

\section{CNNs in medical imaging}
\subsection{What approaches are key components to use deep CNNs in medical imaging applications, especially when data for training is not or sparsely available?}
\subsection{Describe two approaches and the benefit of using them in medical imaging.}

\addtocounter{section}{-1}\section{}\label{sec:todo}