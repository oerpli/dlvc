% !TeX root = ../dlvc.tex
\newcommand{\ttitle}{Image Classification using Linear Models and CNN \TODO}
\hyphenation{}
\maketitle
\keywords{stochastic gradient descent, convolutional neural network}

\begin{abstract}
This is the report for the second assignment of the course ``Deep Learning for Visual Computing'' in the winter term 2016 at the TU Vienna. The task was to build a linear model to classify images of  the CIFAR10 dataset \TODO
\end{abstract}


\section{Linear Models}

\section{Minibatch Gradient Descent}

\section{Preprocessing}

\section{Optimization vs. Machine Learning}

\section{Convolutional Neural Networks}

\begin{figure}[h!t]
\newcommand{\plotref}[1]{{[~\ref{plt:#1}~]}}
\centering
% !TeX root = ../dlvc.tex
\begin{tikzpicture}
\begin{axis}[
	height=160pt
,	xlabel={Epoch}
,	ylabel={Accuracy (\%)}
,	width=1\textwidth
,	xtick={20,40,...,200}
,	ticks=major
,	xmajorgrids
,	y filter/.code={\pgfmathparse{#1*100}\pgfmathresult}
, 	major grid style={line width=0.1pt,draw=gray!30}
,	xmin = 0
,	xmax = 120
,	ymax = 90
%,	ymin=20
%,	ymax=42
,	legend cell align=left
,	legend pos=south east]
%\addplot[plotA]{sin(x)};
%\addplot [clickable coords,LplotD]table[x index=0,y index=1, meta index = 0] {./data/a2/softmaxtinycifar10data.txt};\addlegendentry{Loss}

\addplot [clickable coords,LplotD,uvred]table[x index=0,y index=2, meta index = 0] {./data/a2/cnncifar.txt};\addlegendentry{CNN Training on Cifar10}
\addplot [clickable coords,LplotD,green!80!black]table[x index=0,y index=3, meta index = 0] {./data/a2/cnncifar.txt};\addlegendentry{CNN Validation/Cifar10}

\addplot [clickable coords,LplotA]table[x index=0,y index=2, meta index = 0] {./data/a2/softmaxtinycifar10data.txt};\addlegendentry{SM Training/TC10}

%\addplot [clickable coords,LplotC]table[x index=0,y index=4, meta index = 0] {./data/a2/softmaxtinycifar10data.txt};\addlegendentry{L1 norm, raw vector \ \ }
\addplot [clickable coords,LplotD]table[x index=0,y index=2, meta index = 0] {./data/a2/softmaxhogdata.txt};\addlegendentry{SM Training/HOG}
\addplot [clickable coords,LplotE]table[x index=0,y index=3, meta index = 0] {./data/a2/softmaxhogdata.txt};\addlegendentry{SM Validation/HOG}%\addplot [clickable 

\addplot [clickable coords,LplotB]table[x index=0,y index=3, meta index = 0] {./data/a2/softmaxtinycifar10data.txt};\addlegendentry{SM Validation/TC10}


%\addplot[clickable coords,black!30!red,mark=square,ultra thick,only marks,forget plot]table[x index=0,y index=1, meta index = 0] {./data/a1-test.txt};\label{plt:t1}
%\addplot[clickable coords,black!30!red,mark=pentagon*,thick,only marks,forget plot]table[x index=0,y index=2, meta index = 0] {./data/a1-test.txt};\label{plt:t2}
\end{axis}
\end{tikzpicture}

\caption{Accuracy of kNN classification with neighborhood sizes (k) and L1 and L2 norms on raw image and HOG vectors. Classification on HOG consistently outperforms classification on raw data. Furthermore kNN with L1 distance metric is superior to kNN with L2 distance metric. Performance of best parameters ($k=21, \text{L1 norm}$) on test set with raw \plotref{t2} and HOG data \plotref{t1}. }
\label{figparam}

\end{figure}







